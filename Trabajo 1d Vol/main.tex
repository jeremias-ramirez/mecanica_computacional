\documentclass{article}
\usepackage{amsmath}
\usepackage{amssymb}
\usepackage{graphicx}
\usepackage[utf8]{inputenc}

\title{Trabajo Practico Volúmenes Finitos 1D}
\author{Lagorio Jeremias }
\date{ Octubre 2019}

\begin{document}

\maketitle

\begin{enumerate}
\item 
    \textbf{Discretización de la Ecuación de Conservación escalar }

    Sea $\phi(x,y,z)$ una magnitud escalar, $\Omega$ un dominio en $\mathfrak{R}^3$ y $\Gamma$ su frontera, podemos escribir una ecuación de conservación de esta cantidad en forma diferencial como: 

    \begin{equation}
        \dfrac {{ \partial \phi}} {{ \partial t}} + \nabla \vec{v}\phi+c\phi = \nabla  \frac{k}{\rho C_p} \nabla \phi + Q
        \end{equation}
    Tomando la ecuación diferencial anterior e integrándola en todo el dominio. Si aplico el teorema de Gauss-Green, a los términos que poseen divergencia de flujos estos generan en lugar de integrales de volúmenes, integrales de superficies como se ve en la siguiente ecuación:
    \begin{equation}
       \int_{\Omega} \dfrac {{ \partial \phi}} {{ \partial t}} d\Omega + \int_{\Gamma} \vec{v}\phi d\vec{\Gamma} + \int_{\Omega} c \phi d\Omega = \int_{\Gamma}  \frac{k}{\rho C_p} \nabla \phi \vec{d\Gamma} + \int_{\Omega} Q d\Omega
    \end{equation}
    Juntando las integrales de volumen en un termino y las de superficie en otros.\
    
   \begin{equation}
       \int_{\Omega} (\dfrac {{ \partial \phi}} {{ \partial t}} + c \phi - Q ) d\Omega + \int_{\Gamma} (\vec{v}\phi - \frac{k}{\rho C_p} \nabla \phi) d\vec{\Gamma} = 0
   \end{equation} 
   
    Tomando dominio de calculo y dividirla en porciones denominadas celdas o volúmenes finitos que cubrirán todo el dominio de calculo, además no están superpuestas.
    \begin{equation}
        \sum_{j=1}^{m} \int_{\Omega_j} \Big[ \dfrac {{ \partial \phi}} {{ \partial t}} + c \phi - Q \Big] d\Omega - \int_{\Gamma_j} \Big[ \kappa \nabla \phi- \vec{v}\phi \Big]  \vec{d\Gamma} = 0
    \end{equation}
    Considerando a los volúmenes de la discretización como poliedros, se obtiene entonces:
    \begin{equation}
        \sum_{j=1}^{m} \int_{\Omega_j} \Big[ \dfrac {{ \partial \phi}} {{ \partial t}} + c \phi - Q \Big] d\Omega - \Big\{ \sum_{i=1}^{n} \int_{\Gamma_i} \Big[ \kappa \nabla \phi - \vec{v}\phi \Big] . \vec{d\Gamma_i} \Big\}_j= 0
    \end{equation}
    Considerando los valores de $\phi$ constante por celda y por cara se puede realizar la siguientes aproximaciones
    \begin{equation}
        \begin{aligned}
            \int_{\Omega_j} \phi d\Omega_j = \bar{\phi}V_j \\
            \int_{\Gamma_i} \phi \vec{d\Gamma_i} = \bar{\phi} \vec{S_{fi}}    
        \end{aligned}
    \end{equation}
    Donde $V_j$ es el volumen de la celda j-esima y $\vec{S_{fi}}$ es el área de la frontera con dirección normal a ésta. 
    Quedando la ecuación final
    \begin{equation}
        \sum_{j=1}^{m}  \Big[ \dfrac {{ \partial \phi}} {{ \partial t}} + c \phi - Q \Big] V_j - \Big\{ \sum_{i=1}^{n} \Big[ \kappa \nabla \phi - \vec{v}\phi \Big] . \vec{S_{fi}} \Big\}_j= 0
        \label{eq: volumenFinal}
    \end{equation}
\item 
    \textbf{Aplicación en 1D} 
    
    Analizando ecuación \ref{eq: volumenFinal} con una sola celda P queda:
    \begin{equation}
        \Big[ \dfrac {{ \partial \phi_P}} {{ \partial t}} + c \phi_P - Q_P \Big] V_P - \Big\{ \sum_{i=1}^{n} \Big[ \kappa \nabla \phi - \vec{v}\phi \Big]_{fi} . \vec{S_{fi}} \Big\}_P= 0
        \label{eq: celdaPGeneral}
    \end{equation}
    
    Considerando que la celda P, tiene como vecinos los centroides W y E, con sus respectivas caras w y e.
    El gradiente de las respectivas caras se obtiene a partir de una aproximación centrada. Las aproximaciones son:
    \begin{equation}
        \begin{aligned}
            \nabla \phi_w \vec{S_{fw}}= \frac{\phi_W - \phi_P}{\delta_w}\hat{(-i)}S_w\vec{n_w}\\
            \nabla \phi_w \vec{S_{fe}}= \frac{\phi_E - \phi_P}{\delta_e}\hat{(i)}S_e\vec{n_e}
        \end{aligned}
        \label{eq: gradientes}
    \end{equation} 
    Donde $S_{w|e}$ son la magnitud de las respectivas caras, y $\vec{n_{w|e}}$ son los versores normales a la caras. Las cuales son para 1D $\hat{(-i)} = -1$ y $\hat{(i)} = 1$ respectivamente.
    
    
    Los valores $\delta_{w|e}$ son la distancias entre los centroides W-P y E-P, si se considera una malla uniforme con división dx, ambas distancias son igual a dx.
    El área en ambas caras, con malla uniforme, esta dada por $S_{w|e}= A = dy*t$.
    
    Expandiendo (\ref{eq: celdaPGeneral}) y reemplazando los gradientes por (\ref{eq: gradientes}) , eliminado el termino temporal y agregando todas las consideraciones dichas anteriormente, la ecuación resultante es:
    \begin{multline}
        c \phi_P V_P -  \frac{\phi_W - \phi_P}{\delta_w} A - \vec{v}\phi_{fw} A(-1) \\ 
        - \frac{\phi_E - \phi_P} {\delta_e} A - \vec{v}\phi_{fe} A (1)  = Q_P V_P    
        \label{eq: celdaPGrad}
    \end{multline}
     \begin{enumerate}
         \item 
            \textbf{Esquema Centrado para $\phi_{fi}$} 
            
            Para aproximar $\phi_{fi}$ se utiliza una interpolación lineal entre los centroides que están relacionados por la cara, es decir:
            \begin{equation*}
                \begin{aligned}
                    \phi_w = f_x^w \phi_P + (1 - f_x^w) \phi_W \\
                    \phi_e = f_x^e \phi_P + (1 - f_x^e) \phi_N 
                \end{aligned}      
            \end{equation*} 
            Para una malla uniforme $f_x^{w|e} = 1/2$, quedando las siguientes aproximaciones:
            
            \begin{equation}
                \begin{aligned}
                    \phi_w \vec{S_{fw}}= \frac{\phi_P +  \phi_W}{2} A (-1) \\
                    \phi_e \vec{S_{fe}}= \frac{\phi_P +  \phi_E}{2} A (1) 
                    \label{eq:advectivo}
               \end{aligned}      
            \end{equation} 
    
     \end{enumerate} 

\end{enumerate}
\end{document}