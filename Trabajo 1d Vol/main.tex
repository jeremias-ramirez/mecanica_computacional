\documentclass{article}
\usepackage{amsmath}
\usepackage{amssymb}
\DeclareMathSymbol{\shortminus}{\mathbin}{AMSa}{"39}
\usepackage{graphicx}
\usepackage[utf8]{inputenc}

\title{Trabajo Practico Volúmenes Finitos 1D}
\author{Lagorio Jeremias }
\date{ Octubre 2019}

\begin{document}

\maketitle

\section{Discretización de la Ecuación de Conservación escalar }

    Sea $\phi(x,y,z)$ una magnitud escalar, $\Omega$ un dominio en $\mathbb{R}^3$ y $\Gamma$ su frontera, podemos escribir una ecuación de conservación de esta cantidad en forma diferencial como: 

    \begin{equation}
        \dfrac {{ \partial \phi}} {{ \partial t}} + \nabla \vec{v}\phi+c\phi = \nabla \cdot \left( \frac{k}{\rho C_p} \nabla \phi \right) + Q
        \end{equation}
    Tomando la ecuación diferencial anterior e integrándola en todo el dominio. Si aplico el teorema de Gauss-Green, a los términos que poseen divergencia de flujos estos generan en lugar de integrales de volúmenes, integrales de superficies como se ve en la siguiente ecuación:
    \begin{equation}
       \int_{\Omega} \dfrac { \partial \phi} { \partial t} \, d\Omega + \int_{\Gamma} \vec{v}\phi \cdot \vec{d\Gamma} + \int_{\Omega} c \phi \, d\Omega = \int_{\Gamma}  \frac{k}{\rho C_p} \nabla \phi \cdot \vec{d\Gamma} + \int_{\Omega} Q \, d\Omega
    \end{equation}
    Juntando las integrales de volumen en un termino y las de superficie en otros.\
    
   \begin{equation}
       \int_{\Omega} \left( \dfrac {{ \partial \phi}} {{ \partial t}} + c \phi - Q \right) d\Omega + \int_{\Gamma} \left( \vec{v}\phi - \frac{k}{\rho C_p} \nabla \phi \right) \cdot \vec{d\Gamma} = 0
   \end{equation} 
   
    Tomando dominio de calculo y dividirla en porciones denominadas celdas o volúmenes finitos que cubrirán todo el dominio de calculo, además no están superpuestas.
    \begin{equation}
        \sum_{j=1}^{m} \left[ \int_{\Omega_j} \left( \dfrac {{ \partial \phi}} {{ \partial t}} + c \phi - Q \right) d\Omega - \int_{\Gamma_j} \left( \kappa \nabla \phi- \vec{v}\phi \right) \cdot \vec{d\Gamma} \right] = 0
    \end{equation}
    Considerando a los volúmenes de la discretización como poliedros, se obtiene entonces:
    \begin{equation}
        \sum_{j=1}^{m} \left[ \int_{\Omega_j} \left( \dfrac {{ \partial \phi}} {{ \partial t}} + c \phi - Q \right) d\Omega - \left\{ \sum_{i=1}^{n} \int_{\Gamma_i} \left( \kappa \nabla \phi - \vec{v}\phi \right) \cdot \vec{d\Gamma_i} \right\}_j \right] = 0
    \end{equation}
    Considerando los valores de $\phi$ constante por celda y por cara se puede realizar la siguientes aproximaciones
    \begin{equation}
        \begin{aligned}
            \int_{\Omega_j} \phi \, d\Omega_j = \bar{\phi}V_j \\
            \int_{\Gamma_i} \phi \, \vec{d\Gamma_i} = \bar{\phi} \vec{S_{f_i}}    
        \end{aligned}
    \end{equation}
    Donde $V_j$ es el volumen de la celda j-esima y $\vec{S_{f_i}}$ es el área de la frontera con dirección normal a ésta. 
    Quedando la ecuación final
    \begin{equation}
        \sum_{j=1}^{m} \left[ \left( \dfrac {{ \partial \phi_j}} {{ \partial t}} + c \phi_j - Q \right) V_j - \left\{ \sum_{i=1}^{n} \left( \kappa \nabla \phi - \vec{v}\phi \right)_{f_i} \cdot \vec{S_{fi}} \right\}_j \right] = 0
        \label{eq: volumenFinal}
    \end{equation}
\section{Aplicación en 1D} 
    
    Aplicando la ecuación \ref{eq: volumenFinal} a una sola celda P queda:
    \begin{equation}
        \left( \dfrac { \partial \phi_P} {{ \partial t}} + c \phi_P - Q_P \right) V_P - \left\{ \sum_{i=1}^{n} \left( \kappa \nabla \phi - \vec{v}\phi \right)_{f_i} \cdot \vec{S_{f_i}} \right\}_P= 0
        \label{eq: celdaPGeneral}
    \end{equation}
    
    Considerando que la celda P, tiene como vecinos los centroides W y E, con sus respectivas caras w y e.
    El gradiente de las respectivas caras se obtiene a partir de una aproximación centrada. Las aproximaciones son:
    \begin{equation}
        \begin{aligned}
            \nabla \phi_w \cdot \vec{S_{f_w}} &=
            \frac{\phi_W - \phi_P}{\delta_w} \vec{(\shortminus i)} \cdot \vec{n_w} S_w =
            \frac{\phi_W - \phi_P}{\delta_w} S_w \\
            \nabla \phi_w \cdot \vec{S_{f_e}} &=
            \frac{\phi_E - \phi_P}{\delta_e} \vec{(i)} \cdot \vec{n_e} S_e =
            \frac{\phi_E - \phi_P}{\delta_e} S_e
        \end{aligned}
        \label{eq: gradientes}
    \end{equation} 
    Donde $S_{w|e}$ son la magnitud del área de las respectivas caras, y $\vec{n_{w|e}}$ son los versores normales a la caras. Los cuales son para 1D resultan ser $\vec{(\shortminus i)} = (\shortminus 1, 0, 0)$ e $\vec{(i)} = (1, 0 , 0)$ respectivamente.
    
    
    Los valores $\delta_{w|e}$ son la distancias entre los centroides W-P y E-P, si se considera una malla uniforme con división $dx$, ambas distancias son iguales a $dx$.
    El área en ambas caras, con malla uniforme, esta dada por $S_{w|e}= A = dy*t$.
    
    Expandiendo (\ref{eq: celdaPGeneral}) y reemplazando los gradientes por (\ref{eq: gradientes}) , eliminado el termino temporal y agregando todas las consideraciones dichas anteriormente, la ecuación resultante es:
    
    \begin{multline}
    	    Q_P V_P = c  V_P \phi_P - \frac{\kappa A}{dx} \left( \phi_W - \phi_P \right) + \vec{v}\phi_{f_w} \cdot \vec{S_{f_w}} \\ 
    	        - \frac{\kappa A}{dx} \left( \phi_E - \phi_P \right) + \vec{v}\phi_{f_e} \cdot \vec{S_{f_e}}  
            \label{eq: celdaPGrad}
    \end{multline}
    
    Considerando que
    
    \begin{equation*}
    	\begin{aligned}
	    	\vec{v}\phi_{f_w} \cdot \vec{S_{f_w}} &=
	    	\phi_{f_w} \vec{v} \cdot \vec{n_w} A =
	    	- v_x \phi_{f_w} A
	    	\\
	    	\vec{v}\phi_{f_e} \cdot \vec{S_{f_e}} &=
	    	\phi_{f_e} \vec{v} \cdot \vec{n_e} A =
	    	v_x \phi_{f_e} A
    	\end{aligned}
    \end{equation*}
    
    La ecuación (\ref{eq: celdaPGrad}) puede reducirse, aplicando estas consideraciones y agrupando términos, a:
    \begin{multline}
    	    Q_P V_P = c  V_P \phi_P - \frac{\kappa A}{dx} \left( \phi_W - \phi_P \right) - v_x \phi_{f_w} A \\
    	        - \frac{\kappa A}{dx} \left( \phi_E - \phi_P \right) + v_x \phi_{f_e} A
            \label{eq: celdaPGradVel}
    \end{multline}
    
        \subsection{Esquema Centrado para $\phi_{f_i}$}

        Para aproximar $\phi_{f_i}$ se utiliza una interpolación lineal entre los centroides que están relacionados por la cara, es decir:

        \begin{equation*}
            \begin{aligned}
                \phi_{f_w} &= f_x^w \phi_P + (1 - f_x^w) \phi_W \\
                \phi_{f_e} &= f_x^e \phi_P + (1 - f_x^e) \phi_N
            \end{aligned}
        \end{equation*}

        Para una malla uniforme $f_x^{w|e} = 1/2$, quedando las siguientes aproximaciones:

        \begin{equation*}
            \begin{aligned}
                \phi_{f_w} &= \frac{\phi_P +  \phi_W}{2} \\
                \phi_{f_e} &= \frac{\phi_P +  \phi_E}{2}
			\end{aligned}
        \end{equation*}

        Aplicando estas aproximaciones a la ecuación (\ref{eq: celdaPGradVel}) se obtiene:

        \begin{multline}
    	    Q_P V_P = c  V_P \phi_P - \frac{\kappa A}{dx} \left( \phi_W - \phi_P \right) - v_x \frac{\phi_P +  \phi_W}{2} A \\ 
    	        - \frac{\kappa A}{dx} \left( \phi_E - \phi_P \right) + v_x \frac{\phi_P +  \phi_E}{2} A
    	        \label{eq: celdaPCentrado}
        \end{multline}
            
        Reagrupando los términos, queda la ecuación utilizada para armar el stencil central de la matriz :

        \begin{equation}
			Q_P V_P = \left( c  V_P + 2 \frac{\kappa A}{dx} \right) \phi_P
			+ \left( -\frac{\kappa A}{dx} - \frac{v_x A}{2} \right) \phi_W
			+ \left( -\frac{\kappa A}{dx} + \frac{v_x A}{2} \right) \phi_E
            \label{eq: stencilCentral}
        \end{equation}

\section{Condiciones de bordes}
    \subsection{Condición Dirichlet} 
    \subsubsection{Izquierda (West)}
        En este caso $\phi_W$ no existe y se cambia por el termino del valor dado $\bar{\phi}$. \\
        El término difusivo $\nabla \phi_w$ se cambia por:
        \begin{equation*}
        \frac{\kappa A}{\frac{dx}{2}} \left( \bar{\phi} - \phi_P \right)
        \end{equation*}{}
        El término advectivo $\phi_{fw}$ se cambia por el valor $\bar{\phi}$ directamente, quedando:
        \begin{equation*}
            v_x \bar{\phi} A
        \end{equation*}
        Reemplazando estas ecuaciones en ($\ref{eq: celdaPCentrado}$) se obtiene:
        
        \begin{multline*}
    	    Q_P V_P = c  V_P \phi_P - \frac{\kappa 2 A}{dx} \left( \bar{\phi} - \phi_P \right) - v_x \bar{\phi} A \\ 
    	        - \frac{\kappa A}{dx} \left( \phi_E - \phi_P \right) + v_x \frac{\phi_P +  \phi_E}{2} A
        \end{multline*}
        
        Reagrupando, el stencil para la condición borde es:
        
        \begin{multline}
    	    Q_P V_P + \frac{\kappa 2 A}{dx} \bar{\phi} + v_x \bar{\phi} A = (c  V_P  + \frac{\kappa 3 A}{dx} + \frac{v_x A}{2}) \phi_P 
    	    + ( -\frac{\kappa A}{dx} - \frac{v_x A}{2}) \phi_E
    	    \label{eq:stencilDirW}
        \end{multline}
    \subsubsection{Derecha (East)}   
        Para el caso de la condición en la derecha, el planteno es el mismo, solo que se cambia el término difusivo $\nabla \phi_e$  y el término advectivo $\phi_{fe}$ la ecuación ($\ref{eq: celdaPCentrado}$), resulta:
        \begin{multline*}
    	    Q_P V_P = c  V_P \phi_P - \frac{\kappa A}{dx} \left( \phi_W - \phi_P \right) - v_x \frac{\phi_P +  \phi_W}{2} A \\ 
    	        - \frac{2 \kappa A}{dx} \left( \bar{\phi} - \phi_P \right) + v_x \bar{\phi} A
        \end{multline*}
       El stencil, reagrupando los términos es: 
        \begin{multline}
    	    Q_P V_P + \frac{\kappa 2 A}{dx} \bar{\phi} - v_x \bar{\phi} A = (c  V_P  + \frac{\kappa 3 A}{dx} - \frac{v_x A}{2}) \phi_P 
    	    + ( -\frac{\kappa A}{dx} - \frac{v_x A}{2}) \phi_W
    	    \label{eq:stencilDirE}
        \end{multline}
    \subsection{Condición Neumann}
        La condición de Neumann establece $-\kappa \nabla \phi_{fi} \vec{n} = \textit{q}$ en la dirección de la normal a la cara. Para los términos difusivos $\nabla \phi_{fi}$ el reemplazo es directo, en la cara el cual esta impuesto el flujo. Para los términos advectivos, el valor en la cara se obtiene a partir de utilizar la ecuación de Taylor $\phi_P = \phi_{fi} + \nabla \phi_{fi} \frac{dx}{2} \vec{n_{fi}}$ por lo tanto  $\phi_{fi} = \phi_P - \nabla \phi_{fi} \frac{dx}{2} \vec{n_{fi}} = \phi_P - (-\frac{q}{\kappa}) \frac{dx}{2}$. 
       \\ 
       La ecuación ($\ref{eq: celdaPGeneral}$) se expande y utilizara para hacer los reemplazos necesarios:
     \begin{multline*}
        Q_P V_P = c V_p\phi_P  - \kappa \nabla \phi_{fw}   \vec{n_{f_w}} S \vec{n_{f_w}} + \vec{v}\phi_{fw} S \vec{n_{f_e}} \\
        - \kappa \nabla \phi_{fe} \vec{n_{fe}} S \vec{n_{fe}} + \vec{v}\phi_{fe} S \vec{n_{f_e}}
    \end{multline*}   
        \subsubsection{Izquierda (West)}
        \begin{multline*}
    	    Q_P V_P = c  V_P \phi_P - A q - v_x 
    	    (\phi_P  +  \frac{q}{\kappa} \frac{dx}{2})A \\ 
    	        - \frac{\kappa A}{dx} \left( \phi_E - \phi_P \right) + v_x \frac{\phi_P +  \phi_E}{2} A
        \end{multline*}
        
        El stencil resultante es:
        \begin{multline}
    	    Q_P V_P + A q + v_x A \frac{q}{\kappa} \frac{dx}{2} = (c V_P - v_x A + \frac{\kappa A}{dx} + \frac{v_x A}{2}) \phi_P 
    	    \\ 
    	   + (- \frac{\kappa A}{dx} + \frac{v_x A}{2} )\phi_E
        \end{multline}
        
        \subsubsection{Derecha (East)}
        \begin{multline*}
    	    Q_P V_P = c  V_P \phi_P - \frac{\kappa A}{dx} \left( \phi_W - \phi_P \right) - v_x \frac{\phi_P +  \phi_W}{2} A \\ 
    	        + A q + v_x (\phi_P  +  \frac{q}{\kappa} \frac{dx}{2}) A
    	        \label{eq: celdaPCentrado}
        \end{multline*}
        
        El stencil resultante es:
        \begin{multline}
    	    Q_P V_P - A q - v_x A \frac{q}{\kappa} \frac{dx}{2} = (c V_P + \frac{\kappa A}{dx} - \frac{v_x A}{2} + v_x A) \phi_P 
    	    \\ 
    	   + (- \frac{\kappa A}{dx} - \frac{v_x A}{2} )\phi_W
        \end{multline}
        
\end{document}
